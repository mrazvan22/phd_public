\documentclass[10pt,xcolor=table]{beamer}
 
% \usepackage[utf8]{inputenc}
% \usepackage[T1]{fontenc}
\usepackage[table]{xcolor}    % loads also »colortbl« 
%  \usepackage{enumitem}
\usepackage{ucltemplate}
\usepackage{color}

\definecolor{parCol}{rgb}{0.1, 0.1, 1}
\definecolor{stCol}{rgb}{0.1, 0.6, 0.1}
\definecolor{bothCol}{rgb}{0, 0.5, 0.5}

 
%Information to be included in the title page:
\title{Performance evaluation of disease progression models}
\author{Razvan Valentin Marinescu}
\institute{Center for Medical Image Computing, University College London}
\date{15 November 2016}

% logo of my university
\titlegraphic{\includegraphics[height=1.0cm]{epsrc_logo.jpg}\hspace*{1.75cm}~%
   \includegraphics[height=1.5cm]{NEWpond2017b.png} \hspace*{1.75cm}~ 
   \includegraphics[height=1.0cm]{CDTlogo.png} 
}
 
 
\setbeamersize{text margin left=15pt,text margin right=15pt,text margin bottom=15pt}


\begin{document}
 
\frame{\titlepage}
 
\setbeamerfont{frametitle}{size=\large}


\begin{frame}
\frametitle{Performance evaluation - Aims}

Measure accuracy of: 
\begin{itemize}
  \item \textcolor{parCol}{the fitted model parameters that describe the disease process}
  \item \textcolor{stCol}{predicted stages of subjects}
\end{itemize}

The datasets analysed fall into three broad categories:
\begin{itemize}
 \item simulated datasets: ground truth is user-defined
 \item well-phenotyped datasets: diagnoses are confirmed, stages and the disease process are known
 \begin{itemize}
      \item post-mortem confirmed dementia
      \item familial AD 
      \item prion disease 
 \end{itemize} 
 \item less well-phenotyped datasets: little information on diagnoses, stages or the disease process
 \begin{itemize}
      \item Rotterdam study
      \item ADNI
 \end{itemize} 
 
\end{itemize}

\end{frame}

\begin{frame}
\frametitle{Performance evaluation - measures}

% \setbeamertemplate{itemize items}[square]

% CATEGORIES 
% goodness of fit measures - AIC/BIC, chi-squared
% self-consistency measures - staging consistency
% staging accuracy measures -
% Elapsed time prediction - 
% reproducibility measures - 
% prediction of genetic groups - 
% 
Simulated datasets:
\begin{itemize}
 \item same model used for generating and fitting data:
    \begin{itemize}
    \item \textcolor{parCol}{direct comparison of fitted parameters}
    \end{itemize}
 \item different model:
    \begin{itemize}
    \item need to transform the true parameters then compare
    \end{itemize}

\end{itemize}
\vspace{0.5cm}


Well-phenotyped datasets:
\begin{itemize}
  \item post-mortem confirmed dementia data sets: 
  \begin{itemize}
    \item \textcolor{bothCol}{differential diagnosis}
  \end{itemize}
    
  \item autosomal dominant dementias (AD, HD): 
  \begin{itemize}
    \item \textcolor{bothCol}{prognostic accuracy of later time points from early data}
    \item \textcolor{bothCol}{prediction of genetic groups - supervised or unsupervised}
  \end{itemize}
\end{itemize}

\end{frame}

\begin{frame}
\frametitle{Performance evaluation - measures}

Less well-phenotyped datasets:
\begin{itemize}
 \item \textcolor{parCol}{Goodness of fit measures: AIC, BIC}
 \item \textcolor{stCol}{Staging consistency: follow-up stages $>$ baseline stages}
 \item \textcolor{stCol}{Elapsed time prediction: predict the elapsed time between two visits}
 \item \textcolor{stCol}{Correlation of stages with clinical or leave-out biomarkers:}
   \begin{itemize}
   \item MMSE
   \item CDR-SOB
   \item hippocampal volume
  \end{itemize}
 \item \textcolor{bothCol}{Resampling methods:}
  \begin{itemize}
   \item cross-validation
   \item bootstrapping
  \end{itemize}
 \item \textcolor{bothCol}{Reproducibility for different}: 
 \begin{itemize}
   \item models
   \item datasets
   \item fitting procedures
   \item missing data entries
 \end{itemize}
  
\end{itemize}

\end{frame}

\fontsize{6pt}{7.2}\selectfont
\rowcolors{2}{gray!25}{white}

% \begin{frame}
% \frametitle{Validation methods for disease progression models}
% 
% \begin{tabular}{p{3.3cm} p{3.3cm} p{3.3cm}}
%  Method & Pros & Cons\\
%  \hline
%  AIC/BIC & no extra data required & \\
%  Staging consistency & &\\
%  Elapsed time prediction & &\\ 
%  Diagnosis prediction and differential diagnosis & & diagnosis not precise, required post-mortem confirmed data\\
%  Correlation of stages with clinical markers & & relationship not linear\\
%  Simulations & ground truth available & simulated data might not reflect actual data\\
%  Prediction of leave-out biomarkers & & \\
%  Resampling methods & can also be used to test accuracy of model parameters & computationally intensive\\
%  Reproducibility across models and datasets & & requires different datasets\\
% \end{tabular}
% 
% \end{frame}


 
\end{document}

