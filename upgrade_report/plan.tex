\chapter{Project Plan}
\label{chapter:plan}

In this section we present our future work plan until the end of the PhD. In section \ref{sec:plan_vox} we suggest further modelling that can be done using the voxelwise temporal clustering model presented in chapter \ref{chapter:voxelwise}. In section \ref{sec:plan_perf_eval} we present further analysis that should be done on performance evaluation of disease progression models \ref{sec:plan_perf_eval}. In section \ref{sec:plan_pca_paper_seb} we present further applications for our disease progression models, more precisely on the PCA vs tAD comparison. Finally, in Fig. \ref{fig:plan_gantt} we present a Gantt chart showing the proposed work plan.

\section{Voxelwise Temporal Clustering Model}
\label{sec:plan_vox}

We plan to extend the work done on the voxelwise temporal clustering in several ways and do further analysis. These are summarised in the list below:
\begin{enumerate}
 \item Must do:
 \begin{enumerate}
  \item Run model on amyloid PET imaging from ADNI. This would complement the work currently done using cortical thickness and demonstrate the different types of data which can be analysed by the model. 
  \item Perform more validation on ADNI and DRC datasets. This can include: correlation of stages with cognitive measures, more staging consistency metrics and other goodness of fit measures.
  \item Test a similar model that fits trajectories according to the mean value of vertex measurements in every cluster. We believe that this model gives very similar results but it can run much faster, as the sum over every vertex in equation \ref{eq:dps_vwdpm6} will disappear. 
 \end{enumerate}
 \item Should do:
 \begin{enumerate}
  \item Extensive testing on synthetic data. This can include experiments for many different types of trajectories and number of clusters. For each experiment we plan to plot error-bars showing the error on subject staging and recovered trajectory parameters.
 \end{enumerate}
 \item Might do:
 \begin{enumerate}
  \item Extend model to non-parametric curves such as Gaussian Processes (if time permits). This would involve changing the fundamentals of the model and can be difficult to fine-tune the parameters for the model on each dataset, otherwise the model can easily overfit. 
  \item Extend the model to analyse multimodal data. In order to construct an accurate temporal axis of the disease progression, it can additionally use cognitive tests, CSF biomarkers or combine different voxelwise imaging modalities (cortical thickness + amyloid images). Combining multiple biomarkers is important, as they become abnormal at different stages of the disease. 
  \item Explore other types of voxel dependence, such as spatial correlation functions. 
  \item Extend the model to account fit subgroups of subjects with different clusters of voxels. This will account for disease heterogeneity.
 \end{enumerate}
\end{enumerate}


\section{Performance Evaluation}
\label{sec:plan_perf_eval}

We plan to extend the work on performance evaluation in order to turn this into a comprehensive evaluation of many different metrics (section \ref{sec:plan_metrics}) and disease progression models (section \ref{sec:plan_models}). Based on this work, we also consider getting involved in a prediction challenge using the ADNI dataset. This challenge will involve training some models on current ADNI data and using a subsequent release of ADNI data as the testing set. 

\subsection{Metrics}
\label{sec:plan_metrics}

We plan to further evaluate the following performance metrics or scenarios:
\begin{enumerate}
 \item Must do:
 \begin{enumerate}
 \item Correlation of stages with cognitive measures
 \item Goodness of fit: AIC, BIC
 \item Synthetic data - error in estimating the true parameters
 \end{enumerate}
 \item Should do:
 \begin{enumerate}
 \item Resampling methods: cross-validation, bootstrapping (measure std of estimated parameters)
 \item Robustness to added noise levels in real data
 \end{enumerate}
 \item Might do:
 \begin{enumerate}
  \item Experiment design metrics: these will test which biomarkers are the most informative. This can help make the clinical studies more economical by removing scans or cognitive tests that do not bring in new information.
 \end{enumerate}
\end{enumerate}

\subsection{Models}
\label{sec:plan_models}

We plan to incorporate other fundamentally different models such as:
\begin{enumerate}
 \item Must do:
 \begin{enumerate}
 \item Disease Progression Score (DPS) (Jedynak et al. \cite{jedynak2012computational})
 \item Self-Modelling Regression (SEMOR) (Donohue et al. \cite{donohue2014estimating})
 \end{enumerate}
 \item Might do:
 \begin{enumerate}
 \item Riemannian manifold model (Schiratti et al. \cite{schiratti2015learning})
 \end{enumerate}
Testing other models such as these will help us understand if the assumptions they make are reasonable for the ADNI biomarkers. Moreover, based on the results we will also be able to point out improvements in these models. 
 
 
\end{enumerate}


\subsection{Dataset}
\label{sec:plan_dataset}

Our plan is to do the performance analysis only on the ADNI dataset. We will carefully choose a set of biomarkers derived from MRI (ROI volumes, cortical thickness, atrophy rates), CSF (amyloid-beta, tau), PET (FDG, PIB) and cognitive tests. We need to be careful so that the choice of these biomarkers will not favour certain models. 

\section{PCA vs tAD Analysis}
\label{sec:plan_pca_paper_seb}

We plan to continue the application of our disease progression models to PCA and tAD analysis on the DRC dataset. We suggest to perform the following analysis\footnote{ideas proposed by Dr. Sebastian Crutch}:

\begin{enumerate}
 \item Must do:
\begin{enumerate}
 \item Study rates of atrophy for different PCA subtypes. This analysis will be similar to the one in section \ref{ebm:pca_subgroups}, but will study atrophy rates instead of timing. This can be done with a simple linear mixed effects model or, if there is enough data, with the differential equation model. We would like to test the following hypotheses:
 \begin{enumerate}
  \item Vision subgroup: greater occipital lobe atrophy compared to other subgroups?
  \item Space subgroup: greater superior parietal atrophy compared to other subgroups?
  \item Object subgroup: greater inferior temporal atrophy compared to other subgroups?
 \end{enumerate}
 \item Study the continuum of the three subgroups: build EBM with each subgroup separately, and then try to assign subjects to different models. Plot the normalised likelihoods of belonging to each model on a triangle. 
\end{enumerate}
 \item Should do:
 \begin{enumerate}
 \item Predict progression of cognitive tests from imaging biomarkers at baseline. This task would comprise generating cross-sectional occipital-hippocampal volume discrepancy values at baseline and then test whether different subgroups on this occipital-hippocampal continuum predict different longitudinal patterns of cognitive impairment. More precisely, we hypothesise that patients that have a lower occipital volume at baseline (PCA-like) will have early visual loss with memory spared until later. On the other hand, patients who have a lower hippocampal volume at baseline (tAD-like) will have earlier multi-domain cognitive impairment. 
 \end{enumerate}
 \item Might do:
 \begin{enumerate}
  \item Repeat the same analysis as in 1 (a) but for cortical thickness measures. 
  \item Study the relationship between posterior and anterior grey matter loss. 
  \begin{enumerate}
   \item Based on anatomical subgroups, test if greater inferior posterior atrophy predict greater inferior anterior atrophy and vice versa.
   \item Test if dorsolateral prefrontal lobe atrophy and cortical thinning differ significantly between the three clinical subgroups as follows: (Highest atrophy) Space $>$ Object $>$ Vision (Lowest atrophy). 
   \item Test if inferior prefrontal atrophy and cortical thinning differ significantly between the three subgroups as follows: (Highest atrophy) Object $>$ Space $>$ Vision (Lowest atrophy). 
  \end{enumerate}
  \item Asymmetry analyses: 
  \begin{enumerate}
   \item Measure timing of atrophy: Run the EBM on PCA and tAD using separate biomarkers for left and right hemisphere regions.
   \item Measure rate and extent of atrophy: Run the DEM as above. In addition to that, also plot each left vs right ROI trajectories and measure if there is any statistically significant different in timing, rate and amount of atrophy between regions in the left and right hemisphere. 
  \end{enumerate}
 \end{enumerate}
\end{enumerate}


\tikzset{every picture/.append style={scale=1.6666}}

% Gantt chart
\begin{figure}
\begin{ganttchart}[vgrid, hgrid,y unit chart=0.5cm]{1}{24}
\gantttitle{2017}{12}
\gantttitle{2018}{12}\\
\gantttitlelist{1,...,24}{1}\\
%First Group
\ganttgroup{Voxelwise model}{2}{8} \\
\ganttbar{1a. Amyloid imaging}{2}{3} \\
\ganttbar{1c. "Mean" Model}{3}{3}\\
\ganttbar{2a. Synth. data}{3}{4}\\
\ganttbar{1b. Validation}{4}{5} \\
\ganttbar{Write paper}{4}{7}\\
%\ganttlink{elem0}{elem1}
\ganttlink{elem1}{elem2}
\ganttlink{elem2}{elem3}
\ganttlink{elem3}{elem4}
% \ganttlink{elem4}{elem5}
%\ganttmilestone{Milestone 1}{11}
%Second Group
\ganttgroup{Performance eval.}{8}{22} \\
\ganttbar{Select biomarkers}{8}{8} \\
\ganttbar{1a. Corr. stages}{8}{9} \\
\ganttbar{1b. AIC/BIC}{9}{9} \\
\ganttbar{1c. Synth. data}{9}{11}\\
\ganttbar{2b. Robust. to noise}{11}{13}\\
\ganttbar{2a. Cross-validation}{13}{15}\\
\ganttbar{1a. DPS model}{15}{15}\\
\ganttbar{1b. SEMOR model}{15}{17}\\
\ganttbar{Write paper}{17}{20}\\

\ganttlink{elem7}{elem8}
\ganttlink{elem12}{elem13}
%\ganttmilestone{Milestone 1}{11}
%Third Group
\ganttgroup{PCA vs tAD}{2}{22} \\
\ganttbar{PCA paper write-up}{2}{4}\\
\ganttbar{1a. Atrophy rates}{8}{9}\\
\ganttbar{1b. Subgr. continuum}{11}{12}\\
\ganttbar{2a. Cog. $\rightarrow$ Imaging}{13}{14}\\

\ganttlink{elem17}{elem18}
\ganttlink{elem18}{elem19}
\ganttlink{elem19}{elem20}

\ganttgroup{PhD Thesis}{21}{22} \\

% \ganttlink{elem5}{elem21}
% \ganttlink{elem15}{elem21}

% \ganttbar{Task 1}{26}{28} \\
% \ganttbar{Task 2}{28}{32} \\
% \ganttbar{Task 3}{32}{35}
% %\ganttlink{elem8}{elem9}
% \ganttlink{elem9}{elem10}
% \ganttlink{elem10}{elem11}
%\ganttmilestone{Milestone 1}{11}
\end{ganttchart}
\caption{Gantt chart showing the work schedule for the PhD duration.}
\label{fig:plan_gantt}
\end{figure}


% 
% A fairly complicated example from section 2.9 of the package
% documentation. This reproduces an example from Wikipedia:
% http://en.wikipedia.org/wiki/Gantt_chart
% 
% \definecolor{barblue}{RGB}{153,204,254}
% \definecolor{groupblue}{RGB}{51,102,254}
% \definecolor{linkred}{RGB}{165,0,33}
% \renewcommand\sfdefault{phv}
% \renewcommand\mddefault{mc}
% \renewcommand\bfdefault{bc}
% \setganttlinklabel{s-s}{START-TO-START}
% \setganttlinklabel{f-s}{FINISH-TO-START}
% \setganttlinklabel{f-f}{FINISH-TO-FINISH}
% \sffamily
% \begin{ganttchart}[
%     canvas/.append style={fill=none, draw=black!5, line width=.75pt},
%     hgrid style/.style={draw=black!5, line width=.75pt},
%     vgrid={*1{draw=black!5, line width=.75pt}},
%     today=7,
%     today rule/.style={
%       draw=black!64,
%       dash pattern=on 3.5pt off 4.5pt,
%       line width=1.5pt
%     },
%     today label font=\small\bfseries,
%     title/.style={draw=none, fill=none},
%     title label font=\bfseries\footnotesize,
%     title label node/.append style={below=7pt},
%     include title in canvas=false,
%     bar label font=\mdseries\small\color{black!70},
%     bar label node/.append style={left=2cm},
%     bar/.append style={draw=none, fill=black!63},
%     bar incomplete/.append style={fill=barblue},
%     bar progress label font=\mdseries\footnotesize\color{black!70},
%     group incomplete/.append style={fill=groupblue},
%     group left shift=0,
%     group right shift=0,
%     group height=.5,
%     group peaks tip position=0,
%     group label node/.append style={left=.6cm},
%     group progress label font=\bfseries\small,
%     link/.style={-latex, line width=1.5pt, linkred},
%     link label font=\scriptsize\bfseries,
%     link label node/.append style={below left=-2pt and 0pt}
%   ]{1}{13}
%   \gantttitle[
%     title label node/.append style={below left=7pt and -3pt}
%   ]{WEEKS:\quad1}{1}
%   \gantttitlelist{2,...,13}{1} \\
%   \ganttgroup[progress=57]{WBS 1 Summary}{1}{10} \\
%   \ganttbar[
%     progress=75,
%     name=WBS1A
%   ]{\textbf{WBS 1.1} Activity A}{1}{8} \\
%   \ganttbar[
%     progress=67,
%     name=WBS1B
%   ]{\textbf{WBS 1.2} Activity B}{1}{3} \\
%   \ganttbar[
%     progress=50,
%     name=WBS1C
%   ]{\textbf{WBS 1.3} Activity C}{4}{10} \\
%   \ganttbar[
%     progress=0,
%     name=WBS1D
%   ]{\textbf{WBS 1.4} Activity D}{4}{10} \\[grid]
%   \ganttgroup[progress=0]{WBS 2 Summary Element 2}{4}{10} \\
%   \ganttbar[progress=0]{\textbf{WBS 2.1} Activity E}{4}{5} \\
%   \ganttbar[progress=0]{\textbf{WBS 2.2} Activity F}{6}{8} \\
%   \ganttbar[progress=0]{\textbf{WBS 2.3} Activity G}{9}{10}
%   \ganttlink[link type=s-s]{WBS1A}{WBS1B}
%   \ganttlink[link type=f-s]{WBS1B}{WBS1C}
%   \ganttlink[
%     link type=f-f,
%     link label node/.append style=left
%   ]{WBS1C}{WBS1D}
% \end{ganttchart}
% 
% %
% % A simpler example from the package documentation:
% %
% \begin{ganttchart}{1}{12}
%   \gantttitle{2011}{12} \\
%   \gantttitlelist{1,...,12}{1} \\
%   \ganttgroup{Group 1}{1}{7} \\
%   \ganttbar{Task 1}{1}{2} \\
%   \ganttlinkedbar{Task 2}{3}{7} \ganttnewline
%   \ganttmilestone{Milestone}{7} \ganttnewline
%   \ganttbar{Final Task}{8}{12}
%   \ganttlink{elem2}{elem3}
%   \ganttlink{elem3}{elem4}
% \end{ganttchart}
